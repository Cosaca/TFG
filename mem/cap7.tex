%%%%%%%%%%%%%%%%%%%%%%%%%%%%%%%%%%%%%%%%%%%%%%%%%%%%%%%%%%%%%%%%%%%%%%%%%%%%%
% Chapter 7: Summary and conclusions
%%%%%%%%%%%%%%%%%%%%%%%%%%%%%%%%%%%%%%%%%%%%%%%%%%%%%%%%%%%%%%%%%%%%%%%%%%%%%%%

%++++++++++++++++++++++++++++++++++++++++++++++++++++++++++++++++++++++++++++++

The developed platform, called CodeCharts, aims to make it easier for teachers to visualize the progress of their students, so that once the students have made the challenges, they can see if the courses or workshops are helping in the future promotioning computational thinking when facing activities.

One of the features of this application is that it is designed so that it is only used by the teaching staff, for that reason it has been developed for its use in the most simple and comfortable way. When the teacher has all the students data, he just has to proceed to introduce them on the web, and you can enjoy the functionality that defines it, the statistics.

Regarding the future lines of the project, there are some pending tasks to improve in CodeCharts, since they could not be implemented or developed within the stipulated period.

\begin{itemize}
    \item Creating an account for both teachers and students. As it is currently designed, it is designed so that only teachers are the only one able to gather all the information of the courses. The students could enter with their account to verify in the same way that the teachers do with the statistics of the challenges they have made, so that there is a closer teacher-student interaction.
    \item Data stored in Code.org. As detailed in previous points, the main idea was to implement statistics on the Code.org platform. By failing to do this, the results of the workshops carried out on the Code.org platform could be obtained, so that the teacher can insert them directly into CodeCharts.
    \item Some of the subjects studied in the itinerary of ``Information Technologies'' have contributed to complete the Final Degree Project, which has made easier the development and has allowed me to know new technologies related to the creation of websites.
\end{itemize}

