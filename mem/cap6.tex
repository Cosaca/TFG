%%%%%%%%%%%%%%%%%%%%%%%%%%%%%%%%%%%%%%%%%%%%%%%%%%%%%%%%%%%%%%%%%%%%%%%%%%%%%
% Chapter 6: Conclusiones y líneas futuras
%%%%%%%%%%%%%%%%%%%%%%%%%%%%%%%%%%%%%%%%%%%%%%%%%%%%%%%%%%%%%%%%%%%%%%%%%%%%%%%

%---------------------------------------------------------------------------------

La plataforma desarrollada, llamada CodeCharts, pretende facilitar al profesorado la visualización del progreso de sus alumnos, de manera que una vez que los estudiantes hayan realizado los retos, puedan comprobar en qué medida los cursos
o talleres están ayudando en el fomento del pensamiento computacional a la hora de afrontar las actividades.

Una de las peculiaridades de esta aplicación es que está pensada para que solo la use el personal de la docencia, por ello se ha desarrollado para su uso de la manera más sencilla y cómoda. En el momento que el profesor disponga de todos los datos de los alumnos,
sólo debe proceder a introducirlos en la web, y ya podrá disfrutar de la funcionalidad que lo define, las estadísticas.

Algunas de las asignaturas cursadas en el itinerario de ``Tecnologías de la Información'' han contribuido a la realización del Trabajo Fin de Grado, lo que me ha facilitado su desarrollo y me ha permitido conocer nuevas tecnologías relacionadas con la creación de webs.

En cuanto a las líneas futuras del proyecto, existen algunas tareas pendientes que mejorar en CodeCharts, ya que no se han podido implementar o desarrollar en el plazo estipulado.

\begin{itemize}
    \item Creación de cuenta tanto para los profesores como para los alumnos. Como esta diseñado actualmente está pensado para que solo los profesores sean los encargados de recabar toda la información de los cursos. Los estudiantes podrían entrar
    con su cuenta para comprobar de igual manera que los profesores las estadísticas de los retos que han realizado, de manera que haya una interacción profesor-alumno más cercana.
    \item Datos almacenados en Code.org. Como se detallaba en puntos anteriores, la idea principal era implementar las estadísticas en la plataforma Code.org. Al no conseguirlo, se podrían obtener los resultados de los talleres realizados en la plataforma de Code.org, 
    de manera que el profesor pueda insertarlos directamente en CodeCharts.
\end{itemize}




