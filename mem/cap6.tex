%%%%%%%%%%%%%%%%%%%%%%%%%%%%%%%%%%%%%%%%%%%%%%%%%%%%%%%%%%%%%%%%%%%%%%%%%%%%%
% Chapter 6: Summary and Conlusions
%%%%%%%%%%%%%%%%%%%%%%%%%%%%%%%%%%%%%%%%%%%%%%%%%%%%%%%%%%%%%%%%%%%%%%%%%%%%%%%

%---------------------------------------------------------------------------------

La plataforma desarrollada, llamada CodeCharts, pretende facilitar al profesorado la visualización del progreso de sus alumnos, de manera que una vez que los estudiantes hayan realizado los retos, puedan comprobar en qué medida los cursos
o talleres están ayudando en el fomento del pensamiento computacional a la hora de afrontar las actividades.

Una de las peculiaridades de esta plataforma es que está pensada para que solo la use el personal de la docencia, por ello se ha desarrollado para su uso de la manera más sencilla y cómoda. En el momento que el profesor disponga de todos los datos de los alumnos,
sólo debe proceder a introducirlos en la web, y ya podrá disfrutar de la funcionalidad que lo define, las estadísticas.




